\chapter{Introduction}
\label{introduction}
A finite automaton (FA) is model of computation with applications in different branches of comuter science, e.g., compiler design, formal verification, 
designing of digital ciruits or natural language processing. In formal verificication alone are its uses abundant, 
for example in model checking of safety temporal proprties, abstract regular model checking \cite{armc}, static analysis (metal approach) \cite{metal}, 
or decision procedures of some logics, such as presburger arithmetic or weak 
monadic second-order theory of one successor (WS1S) \cite{mona}.

Many of mentioned applications need to perform certain some \emph{expensive} operations on FA, such as testing universality of on FA (i.e. whether it an
accept any word over a given alphabet), or testing language inclusion of a pair of FA (i.e., testing whether the language of one FA is a subset of the language
of the other FA). \emph{Classical} (so called textbook) approach is based on \emph{complementation} of the language of a FA. Complementation is easy for 
\emph{deterministic} FA (DFA) -- just swapping accepting and non-accepting states -- but a hard problem for \emph{nondeterministic} FA (NFA) with need 
to be determinised first (this may leed to an exponential explosion). Both operations, universality testing and language inclusion over NFA, are PSPACE-complete
problems.

There has been considerable advance in techniques for dealing with these problems. The new techniques are based on the so called \emph{antichains} \cite{antichain}
\cite{congr} or so called \emph{bisimulation upto congruence} \cite{congr}. In general, those techniques do not need explicit construction of the complement
automaton, generate much less states, so construct only sub-automaton, which sufficient for either either proning thah the universality or inclusion hold, or
finding counterexample.

Unfortunately, currently there is no efficient implementation of a general NFA library, that would use the most efficient algoritms for automata operations. The
closest implementation is VATA \cite{libvata}, a general library for nondeterministic finite \emph{tree} automata, which can be used even for NFA (being modelled 
as unary tree automata), but not the optinal performance given by its ability to handle richer structures.
 
\begin{comment}
Finite automata recognize exactly set of regular languages \cite{ullman}.
In formal verification are often problems described as language inclusion \cite{antichain}. Finite automata used for this purposes 
can have very large amount of states, so operations on them could be very computationally demanding.
This fact leads to need of efficient implementation of algorithms for inclusion checking and the other operations on finite automata.

New algorithms for checking language inclusion were introduced \cite{antichain}, \cite{congr}, 
but still not efficiently implemented for finite automata in laguage such as C++. These algorithms are based on pruning out the states, that aren't
unnecessary for checking inclusion, so computation isn't so expensive. Algorithms with optimalization based on antichains and simulation from \cite{antichain}
are implemented in C++ library VATA \cite{libvata}, but only for tree automata. It's possible to use these algorithms for finite automata, which can be represented 
like one dimensional tree automata, but special implementation of algorithms for finite automata will be more efficient. % TODO: Dodat dukaz.
\end{comment}
%TODO: Rozpracovat

The main goal of this work is to extend the library VATA library with efficient implementation of operation for finite automata, not only inclusion, but
also union and intersection. After this introduction, in \ref{teorie}nd chapter of this document, 
will be defined theoretical background. Existing libraries for finite automata manipulation and library VATA will be introduced in chapter \ref{libraries}. 
Design of extentsion for VATA and used algorithms
will take place in chapter \ref{arch}. Implementation and optimalization is possible to find in chapter \ref{implementation}. Evaluation will be
described in chapter \ref{eval} and final conclusion in chapter \ref{concl}.
\chapter{Languages and Finite Automata}
In this chapter we will defined theoretical fundaments for this thesis. No proofs are given, but they can be found in cited literature. 
Firstly, the finite automaton will be defined, then operation for regular languages and the other terms used in this work.
\label{teorie}
\section{Finite automata}
\label{defFA}

	\subsection{Deterministic finite automata}
	\label{defDFA}
	\begin{definition}
		\emph{A Deterministic finite automata} (DFA) is a quintuple $N=(Q,\Sigma,\delta,I,F)$, where
		\begin{itemize}
			\item $Q$ is a finite set of states,
			\item $\Sigma$ is alphabet,
			\item $\delta$ is: $Q \times \Sigma \rightarrow Q,$ we denote $p \xrightarrow{a} q$ transition from $p \in Q$ to $q \in Q$ 
under the input symbol $a \in \Sigma$. 
Function $\delta$ is called transition function.
			\item $I$ is state, that $I \subseteq Q$. Elements of $I$ are called start state.
			\item $F$ is set of states, that $F \subseteq Q$. Elements of $F$ are called final states.
		\end{itemize}
	\end{definition}

	\subsection{Nondeterministic finite automata}
	\label{defNFA}
	\begin{definition}
		\emph{A Nondeterministic finite automata} (NFA) is a quintuple $N=(Q,\Sigma,\delta,I,F)$, where
		\begin{itemize}
			\item $Q$ is a finite set of states,
			\item $\Sigma$ is alphabet,
			\item $\delta$ is: $Q \times \Sigma \rightarrow 2^Q,$ we denote $p \xrightarrow{a} q$ transition from $p \in Q$ to $q \in Q$ 
under the input symbol $a \in \Sigma$. 
Function $\delta$ is called transition function.
			\item $I$ is state, that $I \subseteq Q$. Elements of $I$ are called start states.
			\item $F$ is set of states, that $F \subseteq Q$. Elements of $F$ are called final states.
		\end{itemize}
	\end{definition}

%%%comment
\begin{comment}
	\subsection{Function $\hat{\delta}$}
	\label{defFunct}
		\begin{definition}
		A function $\hat{\delta}$ is define as
		\begin{description}
			$\hat{\delta}: Q\times\Sigma^{*}\rightarrow Q$
		\end{description}

		from $\delta$ by induction on the length of $x$:
			\begin{itemize}
				\item $\hat{\delta}(q,\epsilon)=q$, where $\epsilon$ is empty string
				\item $\hat{\delta}(q,xa)=\delta(\hat{\delta}(q,x),a)$
			\end{itemize}
		\end{definition}
\end{comment}
%%%%		


	\subsection{Run of NFA}
	\label{defRun}
	\begin{definition}
		\emph{A run of NFA }$A$ over a finite word $\omega=\sigma_1\dots\sigma_2$ is a sequence $r=q_0\dots q_n$ of states such that
		\begin{itemize}
			\item $q_0\in I$
			\item $\delta(\sigma_i,q_{i+1})=\sigma_{i+1}$, for all $0\leq i < n$
		\end{itemize}\
		Run is called \emph{accepting} if $q_n\in F$.
	\end{definition}

	\subsection{Accepting word}
	\label{defAWord}
	\begin{definition}
		\emph{Accpeting} word is $\omega \in \Sigma^{*}$, if there exists \emph{accepting} run for $\omega$,
	\end{definition}

	\subsection{Automata union}
	\label{defAUnion}
	\begin{definition}
		$A=(Q_A,\Sigma,\delta_A,I_A,F_A)$ and $B=(Q_B,\Sigma,\delta_B,I_B,F_B)$ are to two NFA's. Their union is defined by
		\begin{description}
			\item $A \cup B=(Q_A\cup Q_B,\Sigma,\delta_A\cup\delta_B,I_A\cup I_B,F_A\cup F_B)$
		\end{description}
	\end{definition}
	
	\subsection{Automata intersection}
	\label{defAInter}
	\begin{definition}
		$A=(Q_A,\Sigma,\delta_A,I_A,F_A)$ and $B=(Q_B,\Sigma,\delta_B,I_B,F_B)$ are to two NFA's. Their intersection is defined by
		\begin{description}
			\item $A \cap B=(Q_A\cap Q_B,\Sigma,\delta,I_A\cap I_B,F_A\cap F_B)$\
		\end{description}
		where $\delta$ is defined by
		\begin{description}
			\item $\{(p_1,p_2) \xrightarrow{a} (q_1,q_2)\ |\ p_1 \xrightarrow{a} p_2 \in \delta_A \wedge q_1 \xrightarrow{a} q_2 \in \delta_B)\}$\
		\end{description}
	\end{definition}

\section{Languages}
		\subsection{Language}
		\label{defLang}
		\begin{definition}
			Let $\Sigma$ be alphabet. \emph{A language }$L$ over $\Sigma$ is $L\subseteq\Sigma^{*}$.
		\end{definition}
		\subsection{Language accepted by automaton $M$}
		\label{defLangM}
		\begin{definition}
			\emph{The set or language accepted by automaton $M$} is the set of all strings accepted by $M$ and is denoted $L(M)$:
			\begin{description*}
				\item $L(M) = \{\omega\in\Sigma^{*}\ |\ \emph{there exists an accepting run for }\omega\emph{ in }L\}$
			\end{description*}
		\end{definition}
		\subsection{Regular language}
		\begin{definition}
			A language $L$ is \emph{regular}, if exists some finite automaton $M$,\\ that $A=L(M)$.
		\end{definition}

		\subsection{Closure properties of regular languages}
		\label{defClose}
		Regular languages are closed under certain operation, if result of this operation on some regular language is regular language too.

		Let introduce the closure properties of regular languages on alphabet $\Sigma$:
		\begin{itemize}
			\item Union:  $L_1 \cup L_2=\{x\ |\ x \in L_1 \vee x\in L_2\}$
			\item Intersection:  $L_1 \cap L_2=\{x\ |\ x \in L_1 \wedge x\in L_2\}$
			\item Complement: $\overline{L}=\{x\ |\ x\not\in L\}$
			\item Difference: $L-K=\{x\ |\ x\in L \wedge x\not\in K\}$
			\item Reversal: $L^R = \{y\ |\ x=a_1\dots a_n \in L \Rightarrow y=a_n\dots a_1\}$
			\item Kleen closure (star): $L^{*}$ is defined as:
			\begin{itemize}
				\item $L_0=\{\epsilon\}$, where epsilon is empty string
				\item $L_1=\{a\}$, where $a\in \Sigma$
				\item $L_{i+1}=\{wv\ |\ w\in V_i \wedge v \in \Sigma\}$, where $i>0$
			\end{itemize}
			\item Concatenation: $LK=\{xy\ |\ x\in L \wedge y\in K\}$
			\item Homomorphism: $h$ is homomorphism on alphabet of language $L$.

$h(L)=\{h(x)\ |\ x \in L\}$
			\item Inverse homomorphism: $h$ is homomorphism on alphabet of language $L$.

$h^{-1}(L)=\{x\ |\ h(x) \in L\}$
		\end{itemize}

	\section{Subset construction}
	\label{subset}
	Now we will define how to construct equivalent DFA $A_{det}$ to given NFA $A=(Q,\Sigma,\delta,S,F)$. 
	This classical ("textbook") approach is called \emph{subset construction}.
	\begin{definition}
	\label{defSubset}
		$A_{det}=(2^Q,\Sigma,\delta_{det},S,F_{det})$, where
		\begin{itemize}
			\item $F_{det}=\{Q'\subseteq Q\ |\ Q'\cap F \not = \varnothing\}$
			\item $\delta_{det}(Q',a)=\bigcup\limits_{q\in Q'}\delta(q,a)$, where $a\in\Sigma$
		\end{itemize}
	\end{definition}
	
\begin{comment}
		\subsection{Union, intersection, inclusion}
\label{defOP}
		\begin{definition}
		Let $L_1$ and $L_2$ be languages. \emph{Union, intersection, inclusion} are defined as:
		\begin{description*}
			\item [$Union:$]  $L_1 \cup L_2=\{x\ |\ x \in L_1 \vee x\in L_2\}$
			\item [$Intersection:$]  $L_1 \cap L_2=\{x\ |\ x \in L_1 \wedge x\in L_2\}$
			\item [$Inclusion:$]  $L_1 \subseteq L_2 \Leftrightarrow L_1 \cap \overline{L_2}=\varnothing$
		\end{description*}
		\end{definition}
		
		\section{Antichain} 
		\begin{definition}
			Let $S$ be \emph{partially order} set. Two elements $a$ and $b$, where $a \in S \wedge b \in S$, are incomparable if $neither\ a \leq b, nor\ b \leq a$.
			\emph{Antichain} is set $A$, where $A \subset S:\ \forall(x,y) \in A$, x and y are incomparable. 
		\end{definition}

		\section{Congruence relation} 
		\begin{definition}
			Let $G$ be group with n-ary operation $O$ and sets of elements $X$. \emph{Congruence} is equivalence relation $R$, which satisfies this condition:
			\begin{description*}
				$a_1 \sim_{R} b_1, a_2 \sim_{R} b_2, a_3 \sim_{R} b_3,\ldots,a_n \sim_{R} b_n \Rightarrow O_n(a_1,a_2,a_3,\ldots,a_n) \sim_{R} O_n(b_1,b_2,b_3,\ldots,b_n)$,
			\end{description*}
			where $a_i \in X, b_i \in X$
		\end{definition}
		%TODO: NAJIT ZDROJ PRO CITOVANI Kongruence a Antichain, idealne jeden zdroj

\section{Simulation}
\label{sim}
		Antichain algorithm is based on pruning out unnecassary states for checking the language inclusion. For this state reduction algorithm 
		needs to compute the simulation on the states of automata.
		\begin{definition}
			A \emph{Simulation} on $\mathcal{A}=(\Sigma,Q,I,F,\delta)$ is relation $\preceq\  \subseteq Q \times Q$ such that $p \preceq r$ only if (i) $p \in F 
			\Rightarrow r \in F$ and (ii) for every transition $p \xrightarrow{a} p'$, there exists transition $r \xrightarrow{a} r'$ such that $p' \preceq r'$ 
		\cite{simulations}.
		\end{definition}
\end{comment}

\chapter{Existing Finite Automata Libraries and VATA} %TODO: zvazit zobecneni na prehled existujicich knihoven
\label{libraries}
There are many different libraries for finite automata. Libraries have various purposes and 
are implemented in different languages. 
At this chapter, some libraries will be described. Described libraries are just examples, which represents typical disadvantages, like
classical approach for language inclusion testing, which needs determinization of automaton. 

Then, VATA library, which implements efficient libraries for \emph{tree} automata, will be introduced.

\section{Existing finite automata libraries}
\label{existinglibraries}
\subsection{dk.brics.automaton}
\label{brics}
\emph{dk.brics.automaton} is estabilished Java package avaliable under BSD license. Last version of this library (1.11-8) was released on September 7th, 2011.
Library can be downloaded and more informations are on \cite{brics}. 

Library can convert can use as input regular expression created by class \emph{RegeExp} to automata.
It supports manipulation with NFA and DFA. Basic operation like union,
intersection, complementation or run of automaton on the given word etc., are avaliable.

Test of language inclusion is also supported, but if the input automaton is NFA, it needs to be converted to DFA. 
This is realised by \emph{subset construction} approach, which is unefficient \cite{cav06}, \cite{antichain}.

\emph{dk.brics.automaton} was ported to antoher two languages in two different libraries, which will be described next.

\subsubsection{libfa}
\emph{libfa}, which is implemented in C. \emph{libfa} is part
of \emph{Augeas} tool. Library is licensed under the LGPL, version 2 or later. It also support both, NFA and DFA. Regular expressions serve like input again.
\emph{libfa} can be found and downloaded on \cite{libfa}.
\emph{libfa} has no explicit operation for inclusion checking, but has operations for intersect and complement of automata, which can serve for inclusion checking.
Main disadvantage of libfa is need of determinization again.

\subsubsection{Fare}
\emph{Fare}, which brings \emph{dk.brics.automaton} from Java to .NET. 
This library has same characteristics like \emph{dk.brics.automaton} or \emph{libfa} and disadvantage in need of determinization is still here. 
\emph{Fare} can be found on \cite{fare}

\subsection{The RWHT FSA toolkit}
The RWHT FSA is toolkit for manipulating finite automata described in \cite{kanthakN04}. The latest version is 0.9.4 from year 2005. Toolkit is written in C++
and avaliable under its special licence, derived from Q Public License v1.0 and the Qt Non-Commercial License v1.0. 

Library supports on-demand computation for better memory efficienty. Another advantage is, that it supports the weighted transitions, so it is possible to use it
for wider range of applications. The RWHT FSA toolkit does not support the explicitly directly, but contains operations for intersection, complement and
determinization, which are unnecessary for inclusion testing. There are used some optimalizations for determinization, but this not compensates need of 
determinization. 

\subsection{Implementation of new efficient algorithms}
New efficient algorithms for inclusion testing, which were introduced in \cite{antichain} and \cite{congr}, was implented for finite automata 
only in OCaml for testing and evaluation purposes. But implementation in C++ could be much more efficient.

The algoritm based on antichains and simulation introduced in \cite{antichain} have been implemented for \emph{tree} automata in library VATA. Description
of this library will be placed in next section.

\begin{comment}
For finite automata manipulation exists many different libraries. Libraries have different purposes and are implemented in different languages, 
e.g. estabilished package of Java \emph{dk.brics.automaton} \cite{brics}, \cite{antichains}, which is also reimplemented for C in \emph{libfa} \cite{libfa} 
and for C\# in \emph{Fare} \cite{fare}. Another libraries are \emph{The RWTH FSA toolkit} \cite{rwth} in C++, which also supports weighted automata, or
\emph{FSA Utilities toolbox} \cite{fsaprolog} in Prolog.

This is not definitely complete enumaration of finite automata libraries, but just few examples. Their main problem is that they do not contain 
the efficient inclusion testing. On the other side, new efficient algorithms introduced in \cite{antichain} and in \cite{congr} have been for finite automata
implemented only in OCaml and implementation in C++ could be much more efficient.

Because of these reasons, algorithms for finite automata manipulation will be implemented as extension of VATA (library for tree automata manipulation) in C++.
\end{comment}

\section{VATA}
\label{VATA}
\subsection{General}
VATA is a highly efficient open source library for manipulating non-deterministic tree automata. 
VATA is implemeted in C++ and uses the Boost C++ library. Informations and download of library can be found on its website 
\footnote{\url{http://www.fit.vutbr.cz/research/groups/verifit/tools/libvata/}} \cite{libvata}. 
\subsection{Design}
VATA provides two kind of encoding for tree automata -- Explicit Encoding (top-down) and Semi-symbolic encoding (top-down and bottom-up). There are supported
these operations: union, intersection, elimination of unreachable states, inclusion checking, computation of simulation relation, language preserving size
reduction based on simulation equivalence. Some operations are able only for specific encoding. Especially for inclusion checking (in both forms -- bottom-up and
top-down) are used advanced optimalization because of improving performance.
More details about implemented operations for tree automata are in \cite{libvata}.
\subsubsection{Explicit Encoding}
For storing explicit encoding top-down transitions is used \emph{hierarchial data structure based on hash tables}. Inserting new transition to this structure 
requires a constant number of steps (exception is worst case scenario). For better performance is used \emph{copy-on-write} technique \cite{libvata}.
\subsubsection{Semi-symbolic Encoding}
Transition functions in semi-symbolic encoding are stored in \emph{multi-terminal binary desicion diagrams} (MTBDD), which are extension of \emph{binary decision 
diagrams}. There are provided top-down and bottom-up representation of tree automata in semi-symbolic encoding \cite{libvata}.
\subsection{Extension for finite automata}
Main goal of this work is create extension of VATA for finite automata in explicit encoding. It is possible to use VATA for finite automata and represents
finite automata like one dimensional, but special implementation will be more efficient. %TODO: (Doplnit nejaky dukaz nebo pozdeji experiment.) 
There will be used implementation of simulation computation from current VATA implementation. %(TODO: Vycet co vsechno pouziju)

\chapter{Design}
\label{arch}

\section{Language inclusion}
\emph{Language inclusion problem} is desicion whether $\mathcal{L(A)} \subseteq \mathcal{L(B)}$, 
where $\mathcal{L(A)}$ is language accepted by non-deterministic automaton $\mathcal{A}$ and so for $\mathcal{B}$. 
This problem is PSPACE-complete \cite{antichain}. For decision of \emph{language inclusion problem} exists some classical algorithms, but their problem
is generating of states, which are not necessary for decision of inclusion problem. This decreases efficient of the algorithm, so it is important to use some
optimalization. In this work are used two approaches for optimalization, first uses antichains and simulations \cite{antichain} for pruning states of NFA 
and second uses congruences \cite{congr}.

For these algorithms let define post-image of production state:\
\begin{definition}
$Post((p,P)):=\{(p',P')|\exists a \in \Sigma: (p,a,p')\in \delta, P'=\{p''|\exists p \in P:(p,a,p'')\in \delta\}\}$
\end{definition}
\subsection{Checking inclusion with antichains}
First approach uses optimalized method based on combination of antichains with simulation from \cite{antichain}. Pseudocode is algorithm \ref{algAntichain}.
\newline
\begin{algorithm}[H]
	\label{algAntichain}
	\KwIn{NFA's $\mathcal{A}=(Q_A,\Sigma,\delta_A,S_A,F_A),\ \mathcal{B}=(Q_B,\Sigma,\delta_B,S_B,F_B)$.\\ 
	 A relation $\preceq \in (\mathcal{A} \cup \mathcal{B})^{\subseteq}$.}
	\KwOut{TRUE if $\mathcal{L(A)}\subseteq\mathcal{L(B)}$. Otherwise, FALSE.}
		\If{there is an accepting product-state in \{$(s,S_{\mathcal{B}})|s\in S_{\mathcal{A}}$\}}
			{return FALSE\;}
		Proccesed:=$\varnothing$\;
		Next:= Initialize(\{$(s,Minimize(S_B))\mid s\in S_A$\})\;
		\While{($Next\neq \varnothing$)}
		{
			Pick and remove a product-state $(r,R)$ from Next and move it to Processed\;
			\ForAll{$(p,P)\in \{(r',Minimize(R'))\mid (r',R')\in Post((r,R))\}$}
			{
				\eIf{$(p,P)$ is an accepting product-state}
				{
					\Return FALSE\;}
					{
						\If{$\not{\exists}p'\in P\ s.t.\ p\preceq p'$}
						{
							\If{$\not{\exists} (x,X) \in Processed\cup Next\ s.t.\ p\preceq x \wedge X\preceq^{\forall \exists}P$}
							{
							 Remove all $(x,X)$ from $Processed\cup Next\ s.t.\ x\preceq p \wedge P\preceq^{\forall \exists}X$\;
							 Add $(p,P)$ to Next\;
							}
						}
				 }
		  }
		}
		\Return TRUE;
	\caption{Language inclusion checking with antichains and simulations}
\end{algorithm}

\subsection{Checking inclusion with congruence}
Checking inclusion with using congruence is based on \emph{Hopcroft and Karp's} algorithm. In fact, algorithm serves for checking language equivalence, but it can
be used also for checking laguage inclusion. Let $X$ and $Y$ be sets of states of NFA's. 
So $\textlbrackdbl X+Y\textrbrackdbl = \textlbrackdbl Y \textrbrackdbl$,
iff $\textlbrackdbl X\textrbrackdbl \subseteq \textlbrackdbl Y \textrbrackdbl$. So it is possible to check equivalence for $X+Y$ and $Y$ \cite{congr}.

Optimalized algorithm uses \emph{congruence closure function} $c$ \cite{congr} for pruning out the unnecessary states for checking equivalence.
Pseudocode is algorithm \ref{algCongr}.\\
%TODO: Popsat, jestli to budu optimalizovat nejak pro inkluzi
\newline
\begin{algorithm}[h]
	\label{algCongr}
	\KwIn{NFA's $\mathcal{A}=(Q_A,\Sigma,\delta_A,s_A,F_A),\ \mathcal{B}=(Q_B,\Sigma,\delta_B,s_B,F_B)$.} 
	\KwOut{TRUE, if $\mathcal{L(A)}$ and $\mathcal{L(B)}$ are in equivalence relation. Otherwise, FALSE.}
		$Processed = \varnothing$\;
		$Next = \varnothing$\;
		insert($s_A$,$s_B$) into $Next$\;
		\While{$Next \neq \varnothing$}
		{
			extract($x$,$y$) from $Next$\;
			\If {$x$,$y \in c(Processed \cup Next)$}
			{skip\;}
			\If{$(x\in F_A \wedge y \notin F_B) \vee (x\notin F_A \wedge y \in F_B)$}
			{
				\Return FALSE\;
			}
			%\ForAll {$a\in \Sigma$}
			%{
			insert($\{(x',y')| (x',y') \in post(x,y)\}$) in $Next$\;
			%}
			insert($x$,$y$) in $Processed$\;
		}
		\Return TRUE;
		\caption{Language equivalence checking with congruence}
\end{algorithm}
\chapter{Implementation}
\label{implementation}
\chapter{Experimental evaluation}
\label{eval}
\chapter{Conclusion}
\label{concl}
